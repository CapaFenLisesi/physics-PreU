\documentclass[main.tex]{subfiles}
%% Current Author: PS
\setcounter{chapter}{15}
\begin{document}
\chapter{Molecular Kinetic Theory}
\begin{content}
\item absolute scale of temperature
\item equation of state
\item kinetic theory of gases
\item kinetic energy of a molecule
\item first law of thermodynamics
\item entropy
\item second law of thermodynamics
\end{content}

\subsection{Candidates should be able to:}

\spec{explain how empirical evidence leads to the gas laws and to the idea of an absolute scale of temperature}



\spec{use the units kelvin and degrees Celsius and convert from one to the other}
\spec{recognise and use the Avogadro number NA = 6.02 × 1023 mol–1}
\spec{recall and use pV = nRT as the equation of state for an ideal gas}
\spec{describe Brownian motion and explain it in terms of the particle model of matter}
\spec{understand that the kinetic theory model is based on the assumptions that the particles occupy no volume,
that all collisions are elastic, and that there are no forces between particles until they collide}
\spec{understand that a model will begin to break down when the assumptions on which it is based are no
longer valid, and explain why this applies to kinetic theory at very high pressures or very high or very low
temperatures}
\spec{derive $PV=\frac{1}{3}Nm<c^2>$ from first principles to illustrate how the microscopic particle model can account
for macroscopic observations}
\spec{recognise and use $\frac{1}{2}m<c^2> = \frac{3}{2}kT$}
\spec{understand and calculate the root mean square speed for particles in a gas}
\spec{understand the concept of internal energy as the sum of potential and kinetic energies of the molecules}
\spec{recall and use the first law of thermodynamics expressed in terms of the change in internal energy, the
heating of the system and the work done on the system}
\spec{recognise and use W = p∆V for the work done on or by a gas}
\spec{understand qualitatively how the random distribution of energies leads to the Boltzmann factor $e^{-\frac{E}{kT}}$ as a
measure of the chance of a high energy}
\spec{apply the Boltzmann factor to activation processes including rate of reaction, current in a semiconductor
and creep in a polymer}
\spec{*describe entropy qualitatively in terms of the dispersal of energy or particles and realise that entropy is
related to the number of ways in which a particular macroscopic state can be realised}
\spec{*recall that the second law of thermodynamics states that the entropy of an isolated system cannot
decrease and appreciate that this is related to probability}
\spec{*understand that the second law provides a thermodynamic arrow of time that distinguishes the future
(higher entropy) from the past (lower entropy)}
\spec{*understand that systems in which entropy decreases (e.g. humans) are not isolated and that when
their interactions with the environment are taken into account their net effect is to increase the entropy
of the Universe}
\spec{*understand that the second law implies that the Universe started in a state of low entropy and that
some physicists think that this implies it was in a state of extremely low probability.}

\end{document}
