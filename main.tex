\documentclass[a4paper,11pt,twoside]{memoir}
\usepackage{subfiles}
\usepackage{graphicx}
\usepackage{titlesec}
\usepackage{tikz}
\usepackage[european,siunitx]{circuitikz}
\usepackage[]{amsmath}
\usepackage{wrapfig}
\usepackage{needspace}
\usepackage{xcolor}
\usepackage{siunitx}

% Should be final package loaded
\usepackage{hyperref}

% Section title formatting
% \titleformat{\chapter}[hang]{\normalfont\Large\bfseries}{\thechapter\ }{1ex}{}
\chapterstyle{section}

% Title page formatting
\pretitle{\begin{center}\Huge\bfseries}
\title{Pre-U Physics Revision Guide}
\posttitle{\vfill\end{center}}
\author{Westminster School}
\predate{\vfill\begin{center}\large}

% Specification command
\newcounter{spec}[chapter]
\newcommand{\spec}[1]{\refstepcounter{spec}\Needspace{5\baselineskip}\textcolor{purple}{\textit{(\alph{spec})~#1}}}
\newcommand{\specstar}[1]{\Needspace{5\baselineskip}\textcolor{purple}{\textit{#1}}}

% Example Question
\newcommand{\answer}{\par \textbf{Answer} \par}
\newsavebox{\examplebox}
\newenvironment{example}
{\begin{lrbox}{\examplebox}\begin{minipage}{0.9\textwidth}\textbf{Example Question}\par}
{\end{minipage}\end{lrbox}\fbox{\usebox{\examplebox}}}

% Content
\newenvironment{content}{\section*{Content}
\begin{itemize}}{\end{itemize}}

% Paragraph formatting
\setlength{\parindent}{0em}
\setlength{\parskip}{1em}
% \setlength{\mathindent}{\parindent}

% Numbering and toc
\setsecnumdepth{chapter}
\setcounter{tocdepth}{0}

\begin{document}
\frontmatter
\begin{titlingpage}
	\maketitle
		\thispagestyle{empty}
	\clearpage
	\section*{About this revision guide}
	This has been written by the Physics Department at Westminster School. It is open source and hosted on Github (\url{https://github.com/mrpsharp/physics-PreU}). Please do submit bug reports, enhancement requests or even pull requests.

	This is very much a revision guide, if you are looking for a textbook try \href{https://openstax.org/details/books/college-physics}{Openstax College Physics} or \emph{Advanced Physics} by Adams and Allday.
\end{titlingpage}

\tableofcontents
\subfile{introduction}

\mainmatter
\part*{Part A}
\addcontentsline{toc}{part}{Part A}
\subfile{1-mechanics}
\subfile{2-gravitational-fields}
\subfile{3-deformation-of-solids}
\subfile{4-energy-concepts}
\subfile{5-electricity}
\subfile{6-waves}
\subfile{7-superposition}
\subfile{8-atomic-and-nuclear-processes}
\subfile{9-quantum-ideas}
\part*{Part B}
\addcontentsline{toc}{part}{Part B}
\subfile{10-rotational-mechanics}
\subfile{11-oscillations}
\subfile{12-electric-fields}
\subfile{13-gravitation}
\subfile{14-electromagnetism}
\subfile{15-special-relativity}
\subfile{16-molecular-kinetic-theory}
\subfile{17-nuclear-physics}
\subfile{18-the-quantum-atom}
\subfile{19-interpreting-quantum-theory}
\subfile{20-astronomy-and-cosmology}
\appendix
\part*{Appendicies}
\addcontentsline{toc}{part}{Appendicies}
\subfile{A-equations}


\backmatter
% bibliography, glossary and index would go here.

\end{document}

%sagemathcloud={"latex_command":"latexmk -xelatex -f -g -bibtex -synctex=1 -interaction=nonstopmode 'main.tex'"}
