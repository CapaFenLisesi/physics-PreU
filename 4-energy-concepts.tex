\documentclass[main.tex]{subfiles}
%% Current Author: 
\setcounter{chapter}{3}
\begin{document}
\chapter{Energy Concepts}
\begin{content}
    \item work
    \item power
    \item potential and kinetic energy
    \item energy conversion and conservation
    \item specific latent heat
\item specific heat capacity
\end{content}

\section*{Candidates should be able to}

\spec{understand and use the concept of work in terms of the product of a force and a displacement in the direction of that force, including situations where the force is not along the line of motion}
\spec{calculate the work done in situations where the force is a function of displacement using the area under a force-displacement graph}
\spec{understand that a heat engine is a device that is supplied with thermal energy and converts some of this energy into useful work}
\spec{calculate power from the rate at which work is done or energy is transferred}
\spec{recall and use $P = Fv$}
\spec{recall and use $\Delta E = mg\Delta h$ for the gravitational potential energy transferred near the Earth’s surface}
\spec{recall and use $g\Delta h$ as change in gravitational potential}
\spec{recall and use $E = \frac{1}{2}Fx$ for the elastic strain energy in a deformed material sample obeying Hooke’s law}
\spec{use the area under a force-extension graph to determine elastic strain energy}
\spec{derive, recall and use $E=\frac{1}{2}kx^2$}
\spec{derive, recall and use $E=\frac{1}{2}mv^2$ for the kinetic energy of a body}
\spec{apply the principle of conservation of energy to solve problems}
\spec{recall and use \[\%\  \text{efficiency} = \frac{\text{useful energy (or power) out}}{\text{total energy (or power) in}} \times 100\]}
\spec{recognise and use $\Delta E = mc \Delta\theta$, where c is the specific heat capacity}
\spec{recognise and use $\Delta E = mL$, where L is the specific latent heat of fusion or of vaporisation}

\end{document}
