\documentclass[main.tex]{subfiles}
%% Current Author: SEQ
\setcounter{chapter}{18}
\begin{document}
\chapter{Interpreting Quantum Theory}

\textbf{Background Information}

(This is not part of the Pre U syllabus but it will help with your understanding.)

\textbf{Wave Equation}

We have already seen how oscillations can be expressed as second order differential equations in Simple Harmonic Motion.

To describe waves we need to go a step further and consider how it changes with time and distance. 

If we imagine ocean waves, if we stand on the end of a jetty, i.e. in a \emph{fixed position}, we can see the water moving up and down as time goes by.

Equally we could take a photograph, i.e. a \emph{fixed time}, and see how the wave changes with distance.

So for a general wave equation we need a second order differential equation for the oscillations but with respect to time, with position kept constant, and with respect to position with time kept constant. To do this we use partial differentiation which is written with a curly $\partial$ and means keep the other variables constant. 

Here is the equation which describes wave motion where u is a function of time and position and c is the speed of the wave.

\[
\frac{\partial^2 u}{\partial t^2} = c^2 \frac{\partial^2 u}{\partial x^2}
\]

The derivation of this is slightly beyond the level of the Pre U although it was first done in the 18$^{\text{th}}$ Century.

\textbf{Schrödinger's Wave Equation}

In the 1920's, Austrian physicist Erwin Schrödinger used de Broglie's idea of particles having wavelengths to try and formulate a wave equation which could explain some of the quantum phenomena which had been observed.

He came up with the following equation originally to describe the energy levels in Hydrogen but it has been incredibly successful (matching experimental data to a high degree of accuracy) in explaining a wide number of quantum mechanical phenomena.

\[
i\hbar \frac{\partial \Psi}{\partial t} = -\frac{\hbar^2}{2m}
\frac{\partial^2 \Psi}{\partial x^2} 
\]

Note the partial differentials and Planck's constant as well as the complex numbers.

You will not need to know this but you need to know that it exists. Also note the wave function $\Psi$ (Psi). It is a function of space and time but when Schrödinger first came up with the equation, he didn't know what it represented.

\spec{*interpret the double-slit experiment using the Copenhagen interpretation (and collapse of the wavefunction), Feynman's sum-over-histories and Everett's many-worlds theory}

The double-slit experiment involved shining a beam of light, electrons or other particles through two small slits and observing the interference pattern cast on a screen. This can be explained using the wave theory, path difference and superposition. (see earlier chapter.)

If the beam of light is reduced in intensity until only one photon passes through the slits at a time, the individual photon will hit the screen at a unique place although it is impossible to predict exactly where. If we continue to send individual photons through we end up with the same diffraction pattern as before.

The problem here is that the diffraction pattern needs a wave interpretation whereas we are sending individual particles through. A particle can't split up between two slits and interfere with itself? 
So when the photon is released it is a particle, as it passes through the slits it is a wave and when it hits the screen it becomes a particle again. 

What causes it to change from one thing to another?

Where is the photon just before it hits the screen?

What is it a wave of?

There hasn't yet been a satisfactory explanation to these questions but here are a few attempts.

\begin{itemize}
\item 
Copenhagen Interpretation

The Danish physicist Niels Bohr set up a conference in Copenhagen with some of the best scientific minds of the time and came up with the first explanation.

Schrodinger had already developed his famous wave equation which had great success at calculating quantum interactions although he didn't establish what the wave function represented. 

It was Max Born who later suggested that the wave function squared (similar to intensity) represented the probability of finding a particle at that point. 

Just before the photon hits the screen it could be anywhere where the wave function isn't zero and yet once it hits the screen and a measurement is made, it is in one precise point on the screen. We call this change from having a probability of being in a number of places to being in one definite location the \emph{wave function collapsing}.

This is the Copenhagen interpretation, until a measurement is made, the particle is simultaneously in all the possible different states and it is only the act of making a measurement which forces it into one unique outcome. 

\item 

Everett's Many World's Theory

A weakness with the Copenhagen interpretation is the question of what is needed to make the wave function collapse? We have slightly skirted around the issue by saying that it is when a measurement is made. But what constitutes a measurement? Do we need a human observer? Particles will undergo many interactions, do they really exist in all possible states until an observation is made?

Everett proposed a solution to this by saying that whenever multiple opportunities occur, for example where our photon hits the screen, the universe splits into multiple versions of itself. Each universe is identical apart from this one difference so one universe will have the photon hitting the centre of the screen whilst in another it will hit the first maximum. When you make a reading the result will depend on which branch of reality you are in. There will be multiple universes with different versions of you in them but you will never meet up. The only reality you will know is what happens in the branch you are in. 

This is great fun to think about, there will be a parallel universe where you have already taken your pre U Physics a year early, scored full marks and are now sipping cocktails on a beach. It also does away with all the arbitrary reasons for the wave function collapsing. The problem with it is the number of universes which are continually being generated. Every time a subatomic particle interacts with another one, the universe splits. Thinking about how many interactions happen every second leads to a mind blowing number of universes and trying to visualise how they can exist alongside each other, perhaps using extra dimensions, is near impossible. 


\item Feynman's sum-over-histories

The last interpretation was proposed by Feynman as a mathematical way of dealing with quantum phenomena. In the case of the double slit experiment we can only know that a photon leaves the laser and where it arrives on the screen after it has been detected. We have no way of knowing which slit it passed through or what route it took to get from the laser to the screen. What Feynman did was to assume that the photon takes \emph{every} possible path. Not just straight lines, not just direct routes. A photon could go to the ends of the universe and back as it makes its way to the screen. If we add all the possible paths together some of them will cancel out if they arrive out of phase, or reinforce if they are in phase. What end up with is a probability of the photon being at any given point on the screen. This probability corresponds exactly with the observed diffraction pattern. 


\end{itemize}
\spec{*describe and explain Schrödinger's cat paradox and appreciate the use of a thought experiment to illustrate and argue about fundamental principles}

Unhappy with the Copenhagen interpretation for the quantum world, Schrödinger set up a now famous thought experiment. The new quantum physics was very counter intuitive in terms of observed physical phenomena and yet gave an incredibly accurate model of the sub atomic world. But the world we see around us is made up of particles so, by extension, the same physics should hold true for both. What Schrödinger did was to take a purely random quantum phenomena and use it to control a macroscopic event.

\textbf{Schrödinger's Cat Paradox}
A cat is put inside a box with a vial of poison and a radioactive material. If there is a radioactive decay, it will break the vial and release the poison and kill the cat. There is a 50/50 chance of the substance decaying in a given time so there is a 50/50 chance of the cat being dead or alive. According to the Copenhagen interpretation, until a measurement is made, both states exist. So before the box is opened, the particle has and hasn't decayed and consequently the cat is simultaneously dead and alive. Schrödinger's use of a dead cat was inspired but he put it forward to show how ridiculous the whole situation was. Later, after it gathered an inordinate amount of attention he regretted ever coming up with it.

In the many world's interpretation, the universe splits into two. One has a box with a decayed particle and a dead cat whilst in the other the cat is alive. When we open the box we see the outcome according to whichever branch of the universe we are in.

\spec{*recognise and use $ \Delta p \Delta x  \geq \frac{h}{2\pi} $ as a form of the Heisenberg uncertainty principle and interpret it}

Imagine two waves, one is a sine wave and the other is a short pulse.

The sine wave has  a definite frequency and wavelength but stretches out indefinitely. The pulse is made up of many sine waves of different frequencies (see Fourier transforms for more on this) but has a fixed size.

So with waves we can see that there is a trade-off between a clearly defined position and wavelength.

We also know that the deBroglie wavelength of a particle is related to the momentum.
\[
\lambda = \dfrac{h}{p}
\]

Mathematically the combination of position and wavelength which gives the lowest combined uncertainty is with a Gaussian (normal) wave and from this we can place a lower bound on the uncertainty in momentum and position.

\textbf{Heisenberg's Uncertainty Principle}

\[
\Delta p \Delta x \geq \dfrac{h}{2 \pi}
\]

so the more accurately we know the momentum of a particle, the less we can know about its position.

note: this is not the effect of taking the measurement which introduces the uncertainty but is an intrinsic limitation due to the wave nature of particles.

\spec{*recognise that the Heisenberg uncertainty principle places limits on our ability to know the state of a system and hence to predict its future}

If we roll a dice we consider it a random process but theoretically if we could measure the exact velocity, spin, gravitational force, air resistance, frictional forces of the table etc. we would know what side the dice would land on.

In the case of quantum particles we can't know the initial conditions precisely because of the limits imposed by Heisenberg's principle. The more accurately we know one thing, the less we know about the other and so we cannot accurately predict what will happen in the future. 

\spec{*recall that Newtonian physics is deterministic, but quantum theory is indeterministic}

With Newtonian physics, the physics we observe in the everyday world, the initial conditions will set into motion a chain of events which determine what happens in the future. If you watch cricket on television you may have seen a computer making leg before wicket decisions by continuing the trajectory of the ball and seeing if it would have hit the wicket had the batsman's leg not been in the way. By knowing the flight of the ball the subsequent path could be determined. 

In quantum physics we cannot know all the initial conditions and so the future cannot be determined.

We say that Newtonian physics is deterministic whilst quantum physics is indeterministic.

\spec{*understand why Einstein thought that quantum theory undermined the nature of reality by being:}
\specstar{(i) indeterministic (initial conditions do not uniquely determine the future)}
\specstar{(ii) non-local (for example, wave-function collapse)}
\specstar{(iii) incomplete (unable to predict precise values for properties of particles).}

Einstein was not happy with quantum theory and couldn't accept that nature could be governed by probability. His famous quote "God does not play dice with the universe" sums up his frustration.

Einstein wanted the universe to be deterministic but we have already seen that this is not what quantum theory predicts.

\textbf{Spukhaft Fernwirkung}

Spooky action at a distance. Einstein was concerned that quantum physics seemed to allow faster than light travel. Going back to the double slit experiment, the instant before the particle hits the screen it could be in a whole range of different places and yet the moment that the measurement is made, the wave function collapses and it is in one unique place. How did it get there so quickly? 

He also proposed another famous thought experiment with Podoslki and Rosen known as the EPR paradox.

If we start with two particles fired with equal velocities from a common origin their subsequent motion can be described using a single, two-particle wave function. After some time a measurement of the position of one of the particles will instantaneously fix the other particle even though it could be a great distance away and was not being measured directly.

The final thing which upset Einstein was not being able to know everything about the state of a particle i.e. incomplete knowledge.
We cannot know the position and momentum simultaneously.

\end{document}


%sagemathcloud={"latex_command":"latexmk -xelatex -f -g -bibtex -synctex=1 -interaction=nonstopmode '19-interpreting-quantum-theory.tex'"}