\documentclass[main.tex]{subfiles}
%% Current Author: SEQ
\setcounter{chapter}{18}
\begin{document}
\chapter{Interpreting Quantum Theory}
\spec{*interpret the double-slit experiment using the Copenhagen interpretation (and collapse of the wavefunction), Feynman's sum-over-histories and Everett's many-worlds theory}

The double-slit experiment involved shining a beam of light, electrons or other particles through two small slits and observing the interference pattern cast on a screen. This can be explained using the wave theory, path difference and superposition. (see earlier chapter.)

If the beam of light is reduced in intensity until only one photon passes through the slits at a time, the individual photon will hit the screen at a unique place although it is impossible to predict exactly where. If we continue to send individual photons through we end up with the same diffraction pattern as before.

The problem here is that the diffraction pattern needs a wave interpretation whereas we are sending individual particles through. A particle can't split up between two slits and interfere with itself? 
So when the photon is released it is a particle, as it passes through the slits it is a wave and when it hits the screen it becomes a particle again. 

What causes it to change from one thing to another?

Where is the photon just before it hits the screen?

What is it a wave of?

There hasn't yet been a satisfactory explanation to these questions but here are a few attempts.

\begin{itemize}
\item 
Copenhagen Interpretation

The Danish physicist Niels Bohr set up a conference in Copenhagen with some of the best scientific minds of the time and came up with the first explanation.

Schrodinger had already developed his famous wave equation which had great success at calculating quantum interactions although he didn't establish what the wave function represented. 

It was Max Born who later suggested that the wave function squared (similar to intensity) represented the probability of finding a particle at that point. 

Before the 




\end{itemize}
\spec{*describe and explain Schrödinger's cat paradox and appreciate the use of a thought experiment to illustrate and argue about fundamental principles}
\spec{*recognise and use $ \Delta p \Delta x  \geq \frac{h}{2\pi} $ as a form of the Heisenberg uncertainty principle and interpret it}
\spec{*recognise that the Heisenberg uncertainty principle places limits on our ability to know the state of a system and hence to predict its future}
\spec{*recall that Newtonian physics is deterministic, but quantum theory is indeterministic}
\spec{*understand why Einstein thought that quantum theory undermined the nature of reality by being:}
\specstar{(i) indeterministic (initial conditions do not uniquely determine the future)}
\specstar{(ii) non-local (for example, wave-function collapse)}
\specstar{(iii) incomplete (unable to predict precise values for properties of particles).}

\end{document}
