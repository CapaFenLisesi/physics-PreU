\documentclass[main.tex]{subfiles}
%% Current Author: SEQ
\setcounter{chapter}{18}
\begin{document}
\chapter{Interpreting Quantum Theory}

\textbf{Background Information}

(This is not part of the Pre U syllabus but it will help with your understanding.)

Wave Equation

We have already seen how oscillations can be expressed as second order differential equations in Simple Harmonic Motion.

To describe waves we need to go a step further and consider how it changes with time and distance. 

If we imagine ocean waves, if we stand on the end of a jetty, i.e. in a \emph{fixed position}, we can see the water moving up and down as time goes by.

Equally we could take a photograph, i.e. a \emph{fixed time}, and see how the wave changes with distance.

So for a general wave equation we need a second order differential equation for the oscillations but with respect to time, with position kept constant, and with respect to position with time kept constant. To do this we use partial differentiation which is written with a curly $\partial$ and means keep the other variables constant. 

Here is the equation which describes wave motion where u is a function of time and position and c is the speed of the wave.

\[
\frac{\partial^2 u}{\partial t^2} = c^2 \frac{\partial^2 u}{\partial x^2}
\]

The derivation of this is slightly beyond the level of the Pre U although it was first done in the 18$^{\text{th}}$ Century.

Schrödinger's Wave Equation

In the 1920's, Austrian physicist Erwin Schrödinger used de Broglie's idea of particles having wavelengths to try and formulate a wave equation which could explain some of the quantum phenomena which had been observed.

He came up with the following equation originally to describe the energy levels in Hydrogen but it has been incredibly successful (matching experimental data to a high degree of accuracy) in explaining a wide number of quantum mechanical phenomena.

\[
i\hbar \frac{\partial \Psi}{\partial t} = -\frac{\hbar^2}{2m}
\frac{\partial^2 \Psi}{\partial x^2} 
\]

Note the partial differentials and Planck's constant as well as the complex numbers.

You will not need to know this but you need to know that it exists. Also note the wave function $\Psi$ (Psi). It is a function of space and time but when Schrödinger first came up with the equation, he didn't know what it represented.

\spec{*interpret the double-slit experiment using the Copenhagen interpretation (and collapse of the wavefunction), Feynman's sum-over-histories and Everett's many-worlds theory}

The double-slit experiment involved shining a beam of light, electrons or other particles through two small slits and observing the interference pattern cast on a screen. This can be explained using the wave theory, path difference and superposition. (see earlier chapter.)

If the beam of light is reduced in intensity until only one photon passes through the slits at a time, the individual photon will hit the screen at a unique place although it is impossible to predict exactly where. If we continue to send individual photons through we end up with the same diffraction pattern as before.

The problem here is that the diffraction pattern needs a wave interpretation whereas we are sending individual particles through. A particle can't split up between two slits and interfere with itself? 
So when the photon is released it is a particle, as it passes through the slits it is a wave and when it hits the screen it becomes a particle again. 

What causes it to change from one thing to another?

Where is the photon just before it hits the screen?

What is it a wave of?

There hasn't yet been a satisfactory explanation to these questions but here are a few attempts.

\begin{itemize}
\item 
Copenhagen Interpretation

The Danish physicist Niels Bohr set up a conference in Copenhagen with some of the best scientific minds of the time and came up with the first explanation.

Schrodinger had already developed his famous wave equation which had great success at calculating quantum interactions although he didn't establish what the wave function represented. 

It was Max Born who later suggested that the wave function squared (similar to intensity) represented the probability of finding a particle at that point. 

Just before the photon hits the screen it could be anywhere where the wave function isn't zero and yet once it hits the screen and a measurement is made, it is in one precise point on the screen. We call this change from having a probability of being in a number of places to being in one definite location the \emph{wave function collapsing}.

This is the Copenhagen interpretation, until a measurement is made, the particle is in all the possible different states and it is the act of making a measurement which forces it into one unique outcome. 




\end{itemize}
\spec{*describe and explain Schrödinger's cat paradox and appreciate the use of a thought experiment to illustrate and argue about fundamental principles}
\spec{*recognise and use $ \Delta p \Delta x  \geq \frac{h}{2\pi} $ as a form of the Heisenberg uncertainty principle and interpret it}
\spec{*recognise that the Heisenberg uncertainty principle places limits on our ability to know the state of a system and hence to predict its future}
\spec{*recall that Newtonian physics is deterministic, but quantum theory is indeterministic}
\spec{*understand why Einstein thought that quantum theory undermined the nature of reality by being:}
\specstar{(i) indeterministic (initial conditions do not uniquely determine the future)}
\specstar{(ii) non-local (for example, wave-function collapse)}
\specstar{(iii) incomplete (unable to predict precise values for properties of particles).}

\end{document}

%sagemathcloud={"latex_command":"latexmk -xelatex -f -g -bibtex -synctex=1 -interaction=nonstopmode '19-interpreting-quantum-theory.tex'"}