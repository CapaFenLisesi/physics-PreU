\documentclass[main.tex]{subfiles}
%% Current Author: PS
\setcounter{chapter}{18}
\begin{document}
\chapter{Interpreting Quantum Theory}
\spec{*interpret the double-slit experiment using the Copenhagen interpretation (and collapse of the wavefunction), Feynman’s sum-over-histories and Everett’s many-worlds theory}

The double-slit experiment involved shining a beam of light through two small slits and observing the pattern cast on a screen.

\spec{*describe and explain Schrödinger’s cat paradox and appreciate the use of a thought experiment to illustrate and argue about fundamental principles}
\spec{*recognise and use $ \Delta p \Delta x  \geq \frac{h}{2\pi} $ as a form of the Heisenberg uncertainty principle and interpret it}
\spec{*recognise that the Heisenberg uncertainty principle places limits on our ability to know the state of a system and hence to predict its future}
\spec{*recall that Newtonian physics is deterministic, but quantum theory is indeterministic}
\spec{*understand why Einstein thought that quantum theory undermined the nature of reality by being:}
\specstar{(i) indeterministic (initial conditions do not uniquely determine the future)}
\specstar{(ii) non-local (for example, wave-function collapse)}
\specstar{(iii) incomplete (unable to predict precise values for properties of particles).}

\end{document}
