\documentclass[main.tex]{subfiles}
%% Current Author: BC
\setcounter{chapter}{5}
\begin{document}
\chapter{Waves}
\begin{content}
\item progressive waves
\item longitudinal and transverse waves
\item electromagnetic spectrum
\item polarisation
\item refraction
\end{content}

\section*{Candidates should be able to:}
\spec{understand and use the terms displacement, amplitude, intensity, frequency, period, speed and wavelength}

\spec{recall and apply $f = \frac{1}{T}$ to a variety of situations not limited to waves}
\spec{recall and use the wave equation $v=f\lambda$}
\spec{recall that a sound wave is a longitudinal wave which can be described in terms of the displacement of molecules or changes in pressure}

When a sound wave travels through a material, the collisions of molecules are parallel to the direction of travel. Energy is transferred through these collisions and the speed of the sound wave will depend on factors such as the density of the material and the temperature. 

When a sounds wave is viewed on an oscilloscope, it looks as though the oscillations are perpendicular to the direction of travel, as in a transverse wave. The y-axis can represent either the displacement of molecules (still in the parallel direction) from their equilibrium position, or the difference in pressure. 

\spec{recall that light waves are transverse electromagnetic waves, and that all electromagnetic waves travel at the same speed in a vacuum}
\spec{recall the major divisions of the electromagnetic spectrum in order of wavelength, and the range of wavelengths of the visible spectrum}

Electromagnetic waves are transverse waves where the oscillations are perpendicular to the direction of travel. In all electromagnetic waves there are actually two waves oscillating perpendicular to each other and to the direction of travel. One is an oscillating magnetic field; the other an oscillation electric field. 

Diagram here! 

The electromagnetic spectrum is the name for the arrangement and classification of electromagnetic waves in order of their wavelengths or frequencies. 

The electromagnetic spectrum is shown below in order of increasing wavelength.

Diagram here!

You can see that the visible light spectrum makes up a small part of the electromagnetic spectrum, with wavelengths between 400 - 700 nm.

\spec{recall that the intensity of a wave is directly proportional to the square of its amplitude}



\spec{use graphs to represent transverse and longitudinal waves, including standing waves}

\emph{Note: Standing waves will be covered in Chapter 7 on Superposition}

There are two types of graphs used to represent transverse and longitudinal waves. You need to be careful as they look similar.

The first graph plots the motion of one part of the wave with time, for example the motion of one water molecule as a water wave goes by. The x-axis on this graph can give you the time period of the wave.

The second graph is a snapshot of a section of the wave at one particular instant in time. On this graph the wavelength can be measured from the x-axis.

\spec{explain what is meant by a plane-polarised wave}

A plane-polarised wave is one where there is only \textbf{one} allowed direction of oscillation. This is only applicable to transverse waves where there are multiple allowed modes of oscillation which are all perpendicular to the direction of travel. A longitudinal wave cannot be polarised as there is already only one direction of oscillation - the direction parallel to that of travel. 

Diagram

All electromagnetic waves can be polarised. 

Consider visible light. Ways of polarisation. 
Uses of polarisation

\spec{recall Malus' Law ($I \propto \cos^2\theta $) and use it to calculate the amplitude and intensity of transmission through a polarising filter}

Malus' Law can be used to work out how the intensity of polarised light changes as it passes through a polaroid filter. The angle $\theta$ is the angle \emph{between} the direction of polarisation of the incident light and the axis of transmission of the polaroid. If you start with unpolarised light, $\theta$ is the angle between the two polaroids.

Intenisty
Amplitude
Example
Third polaroid? 

\spec{recognise and use the expression for refractive index
\[ n = \frac{\sin{\theta_1}}{\sin{\theta_2}} = \frac{v_1}{v_2}\]}

When a wave crosses a boundary which involves a change in speed, refraction occurs.

Diagram.

For light, the refractive index of a medium is the ratio of the speed of light in a vacuum, $c$, to the speed of light in the medium, $v$.
\[n = \frac{c}{v}\]

Therefore the refractive index of a material is always greater than one.

If a wave now crosses a boundary between material 1 and material 2, with the angle of incidence being $\theta_1$ and the angle of refraction being $\theta_2$, the following relationship (Snell's Law) applies:

\[ \frac{n_2}{n_1} = \frac{\sin{\theta_1}}{\sin{\theta_2}}\]

As the refractive index of a material is inversely proportional to the speed of light in that material, we know that

\[\frac{n_2}{n_1} = \frac{v_1}{v_2} \]

Snell's Law now becomes

\[ \frac{n_2}{n_1} = \frac{\sin{\theta_1}}{\sin{\theta_2}} =  \frac{v_1}{v_2}\]

This is the most general form of Snell's Law. For the specific case where material 1 is air we can take $n_1 = 1$ as the speed of light in air is so close to the speed of light in a vacuum. Now, replacing $n_2$ with $n$, the equation is:

\[ n = \frac{\sin{\theta_1}}{\sin{\theta_2}} = \frac{v_1}{v_2}\]

This is the equation given in the specification. Be careful as it only applies to the case where material 1 is air and this might not always be the case.

\spec{derive and recall $sin{c} = \frac{1}{n}$ and use it to solve problems}

If we take Snell's Law for the case where light is travelling from a material of higher refractive index into a material with lower refractive index, $n_1 > n_2$, we know that the light will bend away from the normal with the angle of refraction, $\theta_2$ being larger than the angle of incidence, $\theta_1$. Is 

\spec{recall that optical fibres use total internal reflection to transmit signals}
\spec{recall that, in general, waves are partially transmitted and partially reflected at an interface between media.}
\end{document}
